\documentclass[a4paper, twocolumn]{article}

%%%%%%%%%%%%%%%%%%%%%%%%%%%%%%%%%%%%%%%%%%%%%%%%%%%%%%%%%%%%%%%%%%%%%%%%%%%%%%
%%%%%%%%%  DO NOT EDIT !
%%%%%%%%%  DO NOT EDIT !
%%%%%%%%%  DO NOT EDIT !
%%%%%%%%%  DO NOT EDIT !
%%%%%%%%%  DO NOT EDIT !
%%%%%%%%%  DO NOT EDIT !
%%%%%%%%%%%%%%%%%%%%%%%%%%%%%%%%%%%%%%%%%%%%%%%%%%%%%%%%%%%%%%%%%%%%%%%%%%%%%%

\usepackage[english]{babel}
\usepackage{graphicx}             
\usepackage{tabularx}             
\usepackage{multirow}             
\usepackage{url}                 
\usepackage[ansinew]{inputenc}
\usepackage[small,bf]{caption}   
\usepackage{parskip}
\usepackage{amsmath}             
\usepackage{xcolor}
\usepackage{lipsum} 
 
%%%%%%%%%%%%%%%%%%%%%%%%%%%%%%%%%%%%%%%%%%%%%%%%%%%%%%%%%%%%%%%%%%%%%%%%%%%%%%
% fonts
%%%%%%%%%%%%%%%%%%%%%%%%%%%%%%%%%%%%%%%%%%%%%%%%%%%%%%%%%%%%%%%%%%%%%%%%%%%%%%

\usepackage{mathptmx}


%%%%%%%%%%%%%%%%%%%%%%%%%%%%%%%%%%%%%%%%%%%%%%%%%%%%%%%%%%%%%%%%%%%%%%%%%%%%%%
% spacing and indents
%%%%%%%%%%%%%%%%%%%%%%%%%%%%%%%%%%%%%%%%%%%%%%%%%%%%%%%%%%%%%%%%%%%%%%%%%%%%%%

\setlength{\parskip}{2pt}

%%%%%%%%%%%%%%%%%%%%%%%%%%%%%%%%%%%%%%%%%%%%%%%%%%%%%%%%%%%%%%%%%%%%%%%%%%%%%%
% geometry
%%%%%%%%%%%%%%%%%%%%%%%%%%%%%%%%%%%%%%%%%%%%%%%%%%%%%%%%%%%%%%%%%%%%%%%%%%%%%%

\usepackage[left=1.7cm,right=1.7cm,top=2.5cm,bottom=2.5cm]{geometry}
\setlength{\columnsep}{0.6cm}

%%%%%%%%%%%%%%%%%%%%%%%%%%%%%%%%%%%%%%%%%%%%%%%%%%%%%%%%%%%%%%%%%%%%%%%%%%%%%%
% make a header
%%%%%%%%%%%%%%%%%%%%%%%%%%%%%%%%%%%%%%%%%%%%%%%%%%%%%%%%%%%%%%%%%%%%%%%%%%%%%%

\usepackage{fancyhdr}
 
\pagestyle{fancy} 
\fancyhf{}  
\fancyhead[L]{\small\itshape Proceedings of the $9^{th}$ Conference on Interdisciplinary Musicology -- CIM14.
Berlin, Germany 2014}  
\fancyhead[C]{}  
\fancyhead[R]{}  
\renewcommand{\headrulewidth}{0.0pt}  
\fancyfoot[C]{}  
\renewcommand{\footrulewidth}{0.0pt}  

\pagestyle{fancy}

\makeatletter
\let\ps@plain\ps@fancy 
\makeatother
 
%%%%%%%%%%%%%%%%%%%%%%%%%%%%%%%%%%%%%%%%%%%%%%%%%%%%%%%%%%%%%%%%%%%%%%%%%%%%%%
% adjust titles
%%%%%%%%%%%%%%%%%%%%%%%%%%%%%%%%%%%%%%%%%%%%%%%%%%%%%%%%%%%%%%%%%%%%%%%%%%%%%%

\usepackage{titlesec}
  
\titleformat{\section}{\centering\normalfont\bfseries\sc\fontsize{10}{10}\selectfont}{\thesection.}{0.5em}{}
\titleformat{\subsection}{\normalfont\itshape\fontsize{10}{10}\selectfont}{\thesubsection.}{0.5em}{}
  
\titlespacing{\section}{0pt}{9pt}{2pt}
\titlespacing{\subsection}{0pt}{5pt}{0pt}
 
%%%%%%%%%%%%%%%%%%%%%%%%%%%%%%%%%%%%%%%%%%%%%%%%%%%%%%%%%%%%%%%%%%%%%%%%%%%%%%
% Strongly discourage hyphenation
%%%%%%%%%%%%%%%%%%%%%%%%%%%%%%%%%%%%%%%%%%%%%%%%%%%%%%%%%%%%%%%%%%%%%%%%%%%%%%

\hyphenpenalty=5000
\tolerance=1000

%%%%%%%%%%%%%%%%%%%%%%%%%%%%%%%%%%%%%%%%%%%%%%%%%%%%%%%%%%%%%%%%%%%%%%%%%%%%%%
% hyperref
%%%%%%%%%%%%%%%%%%%%%%%%%%%%%%%%%%%%%%%%%%%%%%%%%%%%%%%%%%%%%%%%%%%%%%%%%%%%%%

\usepackage{hyperref}

\hypersetup{
     unicode=false,           
    pdftoolbar=true,        
    pdfmenubar=true,        
    pdffitwindow=false,      
    pdfstartview={FitH},     
    pdftitle={},     
    pdfauthor={},      
    pdfsubject={},   
    pdfcreator={},   
    pdfproducer={},  
    pdfkeywords={keyword1} {key2} {key3}, 
    pdfnewwindow=true,      
    colorlinks=true,        
    linkcolor=darkgray,          
    citecolor=darkgray,         
    filecolor=darkgray,      
    urlcolor=darkgray            
}

%%%%%%%%%%%%%%%%%%%%%%%%%%%%%%%%%%%%%%%%%%%%%%%%%%%%%%%%%%%%%%%%%%%%%%%%%%%%%%
% abstract settings
%%%%%%%%%%%%%%%%%%%%%%%%%%%%%%%%%%%%%%%%%%%%%%%%%%%%%%%%%%%%%%%%%%%%%%%%%%%%%%
 
\newcommand{\CIMabstract}
[1]
{
\paragraph*{\itshape Abstract:} 
{ \bfseries
\fontsize{8}{9}\selectfont
#1
%
}
\vspace{0.3cm}}

 

  
\begin{document}

\fontsize{9}{9.5}\selectfont
 
%%%%%%%%%%%%%%%%%%%%%%%%%%%%%%%%%%%%%%%%%%%%%%%%%%%%%%%%%%%%%%%%%%%%%%%%%%%%%%
% THE TITLE
%%%%%%%%%%%%%%%%%%%%%%%%%%%%%%%%%%%%%%%%%%%%%%%%%%%%%%%%%%%%%%%%%%%%%%%%%%%%%%

\date{}                     

\title{\vspace{-8mm}\textbf{\sc%
\fontsize{16}{16}\selectfont
%
Haptic pattern representation using music technologies
%
\mbox{}\vspace{-1mm}
%
}}

%%%%%%%%%%%%%%%%%%%%%%%%%%%%%%%%%%%%%%%%%%%%%%%%%%%%%%%%%%%%%%%%%%%%%%%%%%%%%%
% AUTHORS
%%%%%%%%%%%%%%%%%%%%%%%%%%%%%%%%%%%%%%%%%%%%%%%%%%%%%%%%%%%%%%%%%%%%%%%%%%%%%%

\author{ %
%
Adam Tindale$^1$, Michael Cumming$^2$, Sara Diamond$^3$\\
%
 \textit{\normalsize %
$^1$ $^2$ $^3$ OCAD University Toronto, ON M5T 1W1 Canada
}\\
%
\footnotesize 
Correspondence should be addressed to: %
%
%
\href
{mailto:atindale@faculty.ocadu.ca}{atindale@faculty.ocadu.ca},
\href
{mailto:mcumming@ocadu.ca}{mcumming@ocadu.ca},
\href
{mailto:sdiamond@ocadu.ca}{sdiamond@ocadu.ca}
}

\maketitle
%
 
%%%%%%%%%%%%%%%%%%%%%%%%%%%%%%%%%%%%%%%%%%%%%%%%%%%%%%%%%%%%%%%%%%%%%%%%%%%%%%
% ABSTRACT
%%%%%%%%%%%%%%%%%%%%%%%%%%%%%%%%%%%%%%%%%%%%%%%%%%%%%%%%%%%%%%%%%%%%%%%%%%%%%%

\CIMabstract{
Wrist-wearable vibrotactile arrays can serve many functions: typically they are used for non-disruptive notification from social media. They can also be used for direction finding, gaming and entertainment. Authoring and programming of wrist-wearable vibrotactile arrays can be difficult because of the lack of standardized notational systems and file formats. Typically, each implementation of haptic arrays uses bespoke programming, which tends to isolate creative and technical development into non-communicating silos that discourage standardization and sharing. We propose that standard musical notation is an appropriate method for standardization and compositional expressiveness
}.

%%%%%%%%%%%%%%%%%%%%%%%%%%%%%%%%%%%%%%%%%%%%%%%%%%%%%%%%%%%%%%%%%%%%%%%%%%%%%%
% SECTIONS
%%%%%%%%%%%%%%%%%%%%%%%%%%%%%%%%%%%%%%%%%%%%%%%%%%%%%%%%%%%%%%%%%%%%%%%%%%%%%%

\section{Introduction}
This work is related to the design of a wrist-wearable vibrotactile device that combines several functions, including the control of gameplay from a smartphone or tablet, display of user?s heart rate visually and vibrationally and display of interesting visual and vibrational patterns for entertainment and wearer adornment value \cite{tindale2014wearable}. For a device in which vibrotactile patterns play such a major aspect, how to author patterns in a shareable and standardized way becomes an important consideration. 

A wrist-wearable vibrotactile device, or what we call a 'vibe bracelet,' needs activation patterns to operate. It is unclear how to design such patterns in a standardized way using the technology closest at hand, which in our case is Arduino code written in C. The design of vibe patterns for such devices is a specialized, yet multi-disciplinary area that straddles concerns for appropriate design of human computer interfaces and for potentially artistic patterns that look and feel attractive on the wrist. Yet the design of vibrational patterns is not a new issue. Music production, especially that which has a strong rhythmic component, faces a similar issue. Nor does design of rhythmic patterns necessarily involve notational systems. Composers and musicians can easily create musical patterns using digital audio workstation software without first composing these patterns within a system of notation. 

The design of patterns on vibrotactile devices depends on their intended functionality. If they are used primarily as entertainment or adornment devices then this would encourage certain forms of rhythmic composition. If they have other purposes such as way-finding through a space, or notification of game status then they would require quite different pattern-making requirements. 

There are requirements for vibe design on bracelets that seem to transcend the actual use cases you might have for these devices. If one assumes that these devices have multiple tactors (vibe motors), these include the ability to activate a tactor individually (a form of monophony) or activate several tactor simultaneously. Each tactor could then play an individual ?part? in an assemble of parts (polyphony). Other considerations are how to design vibrations of specified durations and intensities, similar to the note lengths and dynamics of musical composition, and how to activate a rhythmic pattern that plays over time. 

Several issues arise: which rhythmic patterns are suitable for vibrotactile devices, the best way of composing these patterns in some standardized way, and how to translate these patterns into a form that the hardware device understands using Midi, or some other protocol.

Which rhythmic patterns are suitable for vibrotactile devices? It depends on:
\begin{itemize}

\item Where the device is placed on the body: receptor density varies greatly between various part of the body. 

\item The function of the device: if vibrations are to lend directionality in a way-finding app, then ones that pulse in a definite direction are appropriate. If creative expression is the goal then any kind of pattern might be appropriate.

\item How many tactors the device has: devices with single tactors can play only simple monophonic patterns. Device with several tactors can play significantly more complex patterns. This is similar to single voice parts compared to assemble of four voice parts or more. The complexity increases enormously going from one part to four, as the history of western polyphony attests. 

\item The kind of information [if any] is being transmitted: encoding of symbolic information can be problematic in tactors, since people are not generally accustomed to decoding symbolic vibrotactile messages. 
\end{itemize}

\section{Related Work}

\subsection{Haptic feedback and communication}
Haptic communication through force feedback is a common technique used in gaming. It purpose is to introduce the sense of touch to interactions such as movement through space, collision with, and proximity to, objects, and notification of awarding of game points and of proximity to dangerous situations. Touch can add sensual aspects and realism to computer interactions and may be useful in reducing sensory overloads from other perceptual modalities such those used in graphical interfaces \cite{oakley2000putting}, or with devices too small to have large visual displays.

Oakley provides a useful taxonomy of haptic-related terms, such as proprioceptive, kinesthetic and tactile. Haptic is a general term relating to the sense of touch, while tactile is more specific and relates to the sensation of pressure rather than that of temperature or pain \cite{oakley2000putting}. The largest organ of the human body is the skin and has great potential for the transmittal of information and sensation \cite{lindeman2006wearable} \cite{brewster2004tactons}. 

Haptic feedback has a long history in the design of computer mice and of other hand controllers \cite{yang2005novel}. Touch is of course useful in generating intimacy in normal social situations \cite{bronner1982haptic}. It can also be used to coordinate action in online or gaming situations \cite{ho1998experiment}. The new Apple Watch also enables wearers to communicate their heartbeat to others wirelessly \cite{johnson2014literature}. 

\subsection{Wearable vibrotactile devices}
Typically, haptic devices beyond experimental contexts are either wearable or holdable. One challenge of wearable devices is how to encourage people to wear them. People tend to be very particular about the look and feel of devices that provide potentially intimate notifications. 

The purposes of such devices are varied and include emulation of attention-getting practices such as the squeezing of wrists, touching of shoulders and pats on arms. As Baumann notes, such social gestures may include a wealth of sub-texts including urgency, affection and reinforcement of social hierarchies \cite{baumann2010emulating}.

Due to their potentially non-disruptive nature, vibrotactile devices could aid in tasks in which the user's attention might be devoted to some other task, such as navigation, or gameplay, or in situations where visual information is scarce or non-existent, as with the blind \cite{ertan1998wearable}.  Devices vary in the degree of body contact they provide. The ability of he body to perceiver vibrotactile stimuli varies greatly between regions of the body dues to variation in skin receptor density \cite{lindeman2006wearable}. Work has been also done in wide-area stimulation instead of areas of the body where receptor density is greatest, such as the tips of the fingers and lips.\cite{lindeman2004towards}.

\subsection{Pattern authoring for vibrotactile devices}
It is clear that appropriate patterns for vibrotactile devices depends on what you want such devices to do. It also depends of course on what the device is capable of doing. 

\subsection{Music notation systems and haptics}
\subsection{Technologies for producing music using notational systems}

\section{Results}
\subsection{Use Cases for Vibrotactile Devices}
One way of creating midi control patterns is to compose music in such a way that midi data describing the music is produced as a side-effect. This, of course, is standard procedure in modern music production. One way of producing midi is by using the text-based music engraving and compositional tool Lilypond.
\section{Discussion}
\section{Future Work}
\section{Conclusion}
\section{Acknowledgements}

%%%%%%%%%%%%%%%%%%%%%%%%%%%%%%%%%%%%%%%%%%%%%%%%%%%%%%%%%%%%%%%%%%%%%%%%%%%%%%
% REFERENCES
%%%%%%%%%%%%%%%%%%%%%%%%%%%%%%%%%%%%%%%%%%%%%%%%%%%%%%%%%%%%%%%%%%%%%%%%%%%%%%

\bibliography{CIM14_bibliography}
\bibliographystyle{CIM14}
  
\end{document}
