\documentclass[a4paper, twocolumn]{article}

%%%%%%%%%%%%%%%%%%%%%%%%%%%%%%%%%%%%%%%%%%%%%%%%%%%%%%%%%%%%%%%%%%%%%%%%%%%%%%
%%%%%%%%%  DO NOT EDIT !
%%%%%%%%%  DO NOT EDIT !
%%%%%%%%%  DO NOT EDIT !
%%%%%%%%%  DO NOT EDIT !
%%%%%%%%%  DO NOT EDIT !
%%%%%%%%%  DO NOT EDIT !
%%%%%%%%%%%%%%%%%%%%%%%%%%%%%%%%%%%%%%%%%%%%%%%%%%%%%%%%%%%%%%%%%%%%%%%%%%%%%%

\usepackage[english]{babel}
\usepackage{graphicx}             
\usepackage{tabularx}             
\usepackage{multirow}             
\usepackage{url}                 
\usepackage[ansinew]{inputenc}
\usepackage[small,bf]{caption}   
\usepackage{parskip}
\usepackage{amsmath}             
\usepackage{xcolor}
\usepackage{lipsum} 
 
%%%%%%%%%%%%%%%%%%%%%%%%%%%%%%%%%%%%%%%%%%%%%%%%%%%%%%%%%%%%%%%%%%%%%%%%%%%%%%
% fonts
%%%%%%%%%%%%%%%%%%%%%%%%%%%%%%%%%%%%%%%%%%%%%%%%%%%%%%%%%%%%%%%%%%%%%%%%%%%%%%

\usepackage{mathptmx}


%%%%%%%%%%%%%%%%%%%%%%%%%%%%%%%%%%%%%%%%%%%%%%%%%%%%%%%%%%%%%%%%%%%%%%%%%%%%%%
% spacing and indents
%%%%%%%%%%%%%%%%%%%%%%%%%%%%%%%%%%%%%%%%%%%%%%%%%%%%%%%%%%%%%%%%%%%%%%%%%%%%%%

\setlength{\parskip}{2pt}

%%%%%%%%%%%%%%%%%%%%%%%%%%%%%%%%%%%%%%%%%%%%%%%%%%%%%%%%%%%%%%%%%%%%%%%%%%%%%%
% geometry
%%%%%%%%%%%%%%%%%%%%%%%%%%%%%%%%%%%%%%%%%%%%%%%%%%%%%%%%%%%%%%%%%%%%%%%%%%%%%%

\usepackage[left=1.7cm,right=1.7cm,top=2.5cm,bottom=2.5cm]{geometry}
\setlength{\columnsep}{0.6cm}

%%%%%%%%%%%%%%%%%%%%%%%%%%%%%%%%%%%%%%%%%%%%%%%%%%%%%%%%%%%%%%%%%%%%%%%%%%%%%%
% make a header
%%%%%%%%%%%%%%%%%%%%%%%%%%%%%%%%%%%%%%%%%%%%%%%%%%%%%%%%%%%%%%%%%%%%%%%%%%%%%%

\usepackage{fancyhdr}
 
\pagestyle{fancy} 
\fancyhf{}  
\fancyhead[L]{\small\itshape Proceedings of the $9^{th}$ Conference on Interdisciplinary Musicology -- CIM14.
Berlin, Germany 2014}  
\fancyhead[C]{}  
\fancyhead[R]{}  
\renewcommand{\headrulewidth}{0.0pt}  
\fancyfoot[C]{}  
\renewcommand{\footrulewidth}{0.0pt}  

\pagestyle{fancy}

\makeatletter
\let\ps@plain\ps@fancy 
\makeatother
 
%%%%%%%%%%%%%%%%%%%%%%%%%%%%%%%%%%%%%%%%%%%%%%%%%%%%%%%%%%%%%%%%%%%%%%%%%%%%%%
% adjust titles
%%%%%%%%%%%%%%%%%%%%%%%%%%%%%%%%%%%%%%%%%%%%%%%%%%%%%%%%%%%%%%%%%%%%%%%%%%%%%%

\usepackage{titlesec}
  
\titleformat{\section}{\centering\normalfont\bfseries\sc\fontsize{10}{10}\selectfont}{\thesection.}{0.5em}{}
\titleformat{\subsection}{\normalfont\itshape\fontsize{10}{10}\selectfont}{\thesubsection.}{0.5em}{}
  
\titlespacing{\section}{0pt}{9pt}{2pt}
\titlespacing{\subsection}{0pt}{5pt}{0pt}
 
%%%%%%%%%%%%%%%%%%%%%%%%%%%%%%%%%%%%%%%%%%%%%%%%%%%%%%%%%%%%%%%%%%%%%%%%%%%%%%
% Strongly discourage hyphenation
%%%%%%%%%%%%%%%%%%%%%%%%%%%%%%%%%%%%%%%%%%%%%%%%%%%%%%%%%%%%%%%%%%%%%%%%%%%%%%

\hyphenpenalty=5000
\tolerance=1000

%%%%%%%%%%%%%%%%%%%%%%%%%%%%%%%%%%%%%%%%%%%%%%%%%%%%%%%%%%%%%%%%%%%%%%%%%%%%%%
% hyperref
%%%%%%%%%%%%%%%%%%%%%%%%%%%%%%%%%%%%%%%%%%%%%%%%%%%%%%%%%%%%%%%%%%%%%%%%%%%%%%

\usepackage{hyperref}

\hypersetup{
     unicode=false,           
    pdftoolbar=true,        
    pdfmenubar=true,        
    pdffitwindow=false,      
    pdfstartview={FitH},     
    pdftitle={},     
    pdfauthor={},      
    pdfsubject={},   
    pdfcreator={},   
    pdfproducer={},  
    pdfkeywords={keyword1} {key2} {key3}, 
    pdfnewwindow=true,      
    colorlinks=true,        
    linkcolor=darkgray,          
    citecolor=darkgray,         
    filecolor=darkgray,      
    urlcolor=darkgray            
}

%%%%%%%%%%%%%%%%%%%%%%%%%%%%%%%%%%%%%%%%%%%%%%%%%%%%%%%%%%%%%%%%%%%%%%%%%%%%%%
% abstract settings
%%%%%%%%%%%%%%%%%%%%%%%%%%%%%%%%%%%%%%%%%%%%%%%%%%%%%%%%%%%%%%%%%%%%%%%%%%%%%%
 
\newcommand{\CIMabstract}
[1]
{
\paragraph*{\itshape Abstract:} 
{ \bfseries
\fontsize{8}{9}\selectfont
#1
%
}
\vspace{0.3cm}}

 

  
\begin{document}

\fontsize{9}{9.5}\selectfont
 
%%%%%%%%%%%%%%%%%%%%%%%%%%%%%%%%%%%%%%%%%%%%%%%%%%%%%%%%%%%%%%%%%%%%%%%%%%%%%%
% THE TITLE
%%%%%%%%%%%%%%%%%%%%%%%%%%%%%%%%%%%%%%%%%%%%%%%%%%%%%%%%%%%%%%%%%%%%%%%%%%%%%%

\date{}                     

\title{\vspace{-8mm}\textbf{\sc%
\fontsize{16}{16}\selectfont
%
Haptic pattern representation using music technologies
%
\mbox{}\vspace{-1mm}
%
}}

%%%%%%%%%%%%%%%%%%%%%%%%%%%%%%%%%%%%%%%%%%%%%%%%%%%%%%%%%%%%%%%%%%%%%%%%%%%%%%
% AUTHORS
%%%%%%%%%%%%%%%%%%%%%%%%%%%%%%%%%%%%%%%%%%%%%%%%%%%%%%%%%%%%%%%%%%%%%%%%%%%%%%

\author{ %
%
Adam Tindale$^1$, Michael Cumming$^2$, Sara Diamond$^3$\\
%
 \textit{\normalsize %
$^1$ $^2$ $^3$ OCAD University Toronto, ON M5T 1W1 Canada
}\\
%
\footnotesize 
Correspondence should be addressed to: %
%
% MAKE SURE TO CHANGE BOTH ENTRIES !!!!
%
\href
{mailto:atindale@faculty.ocadu.ca}{atindale@faculty.ocadu.ca},
\href
{mailto:mcumming@ocadu.ca}{mcumming@ocadu.ca},
\href
{mailto:sdiamond@ocadu.ca}{sdiamond@ocadu.ca}
}

\maketitle
%
 
%%%%%%%%%%%%%%%%%%%%%%%%%%%%%%%%%%%%%%%%%%%%%%%%%%%%%%%%%%%%%%%%%%%%%%%%%%%%%%
% ABSTRACT
%%%%%%%%%%%%%%%%%%%%%%%%%%%%%%%%%%%%%%%%%%%%%%%%%%%%%%%%%%%%%%%%%%%%%%%%%%%%%%

\CIMabstract{
Wrist-wearable vibrotactile arrays can serve many functions: typically they are used for non-disruptive notification from social media. They can also be used for direction finding, gaming and entertainment. Authoring and programming of wrist-wearable vibrotactile arrays can be difficult because of the lack of a standardized notational system and standard file formats. Typically, each implementation of haptic arrays uses bespoke programming, which tends to isolate creative and technical development into non-communicating silos and discourage standardization and sharing. We propose that standard musical notation is an appropriate method for standardization and compositional expressiveness}.

%%%%%%%%%%%%%%%%%%%%%%%%%%%%%%%%%%%%%%%%%%%%%%%%%%%%%%%%%%%%%%%%%%%%%%%%%%%%%%
% SECTIONS
%%%%%%%%%%%%%%%%%%%%%%%%%%%%%%%%%%%%%%%%%%%%%%%%%%%%%%%%%%%%%%%%%%%%%%%%%%%%%%

\section{Introduction}
This work is related to the design of a wrist-wearable vibrotactile device that combines several functions, including the control of gameplay from a smartphone or tablet, display of user?s heart rate visually and vibrationally and display of interesting visual and vibrational patterns for entertainment and wearer adornment value \cite{tindale2014wearable}. For a device in which vibrotactile patterns play such a major aspect, how to author patterns in a shareable and standardized way becomes an important consideration. 
\section{Related Work}

\subsection{Haptic feedback and communication}
Haptic communication through force feedback is a common technique used in gaming. It purpose is to introduce the sense of touch to interactions such as movement through space, collision with, and proximity to, objects, and notification of awarding of game points and of proximity to dangerous situations. Touch can add sensual aspects and realism to computer interactions and may be useful in reducing sensory overloads from other perceptual modalities such those used in graphical interfaces \cite{oakley2000putting}. Oakley provides a useful taxonomy of haptic-related terms, such as proprioceptive, kinesthetic and tactile. Haptic is a general term relating to the sense of touch, while tactile is more specific and relates to the sensation of pressure rather than that of temperature or pain \cite{oakley2000putting}.

Haptic feedback has a long history in the design of computer mice and of other hand controllers \cite{yang2005novel}. Touch is of course useful in generating intimacy in normal social situations \cite{bronner1982haptic}. It can also be used to coordinate action in online or gaming situations \cite{ho1998experiment}. The new Apple Watch also enables wearers to communicate their heartbeat to others wirelessly \cite{johnson2014literature}. 

\subsection{Wearable vibrotactile devices}



\subsection{Pattern authoring for vibrotactile devices}
\subsection{Music notation systems and haptics}
\subsection{Technologies for producing music using notational systems}

\section{Results}
\section{Discussion}
\section{Future Work}
\section{Conclusion}
\section{Acknowledgements}

%%%%%%%%%%%%%%%%%%%%%%%%%%%%%%%%%%%%%%%%%%%%%%%%%%%%%%%%%%%%%%%%%%%%%%%%%%%%%%
% REFERENCES
%%%%%%%%%%%%%%%%%%%%%%%%%%%%%%%%%%%%%%%%%%%%%%%%%%%%%%%%%%%%%%%%%%%%%%%%%%%%%%

\bibliography{CIM14_bibliography}
\bibliographystyle{CIM14}
  
\end{document}
